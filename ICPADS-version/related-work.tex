\section{Related Work}
\label{s-related-work}
It has been theoretically proven that purely peer-to-peer blockchains are probabilistic state machines that cannot provide any finality guarantee \cite{7756226}. The proof follows easily from the fact that malicious peers can reach $51\%$ majority in a network and reverse all transactions. So any discussion of blockchain ledger stability assumes that the majority mining power is held by the honest peers. A more practical analysis \cite{10.1007/978-3-319-56614-6_22} shows that assuming a bounded network delay and no message loss, a blockchain network achieves consistency (i.e., transaction finality) with probability approaching $1$. The problem is messages can be lost and the network delay bound cannot be predetermined in practical peer-to-peer network even with majority honest peers. Hence, \textit{probabilistic} transaction finality is the best that can be achieved without any new restriction on the blockchain network.

However, existing blockchain consensus protocols that claim to provide transaction finality impose restrictions on network participation that are difficult to realize in a peer-to-peer setting where participants can join and leave at any time. For example, the Ripple \cite{David2014TheRP} protocol assumes that $80\%$ neighbors of each honest peer are honest and malicious neighbors cannot intercept communications between honest peers. Similarly, the Stellar consensus protocol \cite{thestellar} assumes that the majority honest peers of the network form a connected sub-network that provides guaranteed message delivery. The Tendermint consensus protocol \cite{Buchman2018TheLG} also assumes $\frac{2}{3}$ majority of honest peers receives every message within a bounded time delay. Under this assumption, it provides a \textit{PBFT} \cite{Castro:1999:PBF:296806.296824} solution decentralized for blockchain technology domain. The central issue with all these protocols is that they assume a gossip communication protocol \cite{gossip} can be utilized as a mean to implement a reliable broadcast primitive \cite{Jalote:1994:FTD:179250} in a peer-to-peer network. This assumption is not theoretically valid. Hence these protocols suffer stagnation when broadcast fails.          

The forerunner of the \textit{business blockchain movement}, Hyperledger, provides transaction finality for two of its blockchain solutions: \textit{Iroha} \cite{yac} and \textit{Fabric} \cite{patent:20180150799}. Both assume there is a BFT ordering service that orders and reliably delivers messages to all network peers. The network is definitely not peer-to-peer in either case and so far no convincing BFT solution for a distributed ordering service has been proposed. 

None of the aforementioned protocols work on a network incentivised by PoW mining. The only notable transaction finality solution for PoW blockchain networks is Ethereum's Casper \cite{casper}. Casper assigns validator nodes (based on deposit of stakes) among the mining population for regularly voting on alternative blockchain versions. The version receiving $\frac{2}{3}$ majority of votes is called a finalized checkpoint, thus irreversible. Since the underlying networking characteristics remain the same, $\frac{2}{3}$ validator nodes may fail to collaborate for establishing a voting consensus. Thus, in the worst case, no checkpoint can be finalized.                

A major difference between ours and existing transaction finality solutions is that we do not assume the existence of any reliable broadcast primitive in the underlying peer-to-peer network. Still, we periodically establish checkpoints in the blockchain ledger. In addition, although we employ voting for reaching consensus about a checkpoint like others, our voting process is also governed by a PoW incentive unlike any. 