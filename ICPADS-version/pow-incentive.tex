\section{An Analysis of PoW Incentive}
\label{s-incentive}
From a technical perspective, Nakamoto's Bitcoin \cite{bitcoin} took the world by storm because it shows for the first time that a highly secure distributed system can be built from laissez-faire collaboration of a network of mutually-distrusted autonomous peers. Even continuous network connectivity among the peers are not required and the network-wide connection topology remains dynamic. He deemed a PoW for block mining is necessary to thwart traditional networking attacks such as the Sybil Attack \cite{Douceur:2002:SA:646334.687813} on such a system. To successfully alter a shared blockchain ledger of honest miners, an attacker has to surpass their combined mining power to construct an alternative chain that is longer; then offer it to them for synchronization. The effort becomes more futile as the population of honest peers increases.

The strength of information security in a blockchain network is proportional to the mining capacity of its honest peers. Therefore, from its inception, keeping honest peers interested in network participation is a central concern in blockchain technology. Nakamoto's ingenious idea was to make participation in block mining process an economic activity \cite{Kroll2013TheEO} by rewarding a block miner for his/her PoW in terms of transaction fees and a block reward. This is the crux of the PoW incentive scheme.

PoW incentive makes honest behavior the rational behavior \cite{bitcoin} in a blockchain network governed by economic principles. This particular feat is the source of its resilience. So far no alternative to PoW incentive is proven to be equally scalable, secure, and censorship-resistant. Consequently, the largest and most-powerful blockchain networks are still PoW networks despite widespread scrutiny of PoW mining's power consumption cost \cite{powcost}. Hence, transaction finality solutions should also target PoW mining based blockchain networks.

In the PoW incentive scheme, all miner activities including computation and communication are governed by rational behavior of maximizing economic gain. As a result, in such a network a consensus about an irreversible ledger state (i.e., transaction finality) cannot be established without halting the mining process, as explained in the following lemma:

\begin{lemma}
\label{l-halt}
Transaction finality is unachievable without halting block mining in a PoW blockchain network.  
\end{lemma}

\begin{proof}
Since there is no benefit for a peer in throwing away its local blockchain ledger version and accepting a neighbor's blockchain that has the same amount of PoW being done on it, information about alternative equally good blockchain versions will not spread in the network. Consequently, at any instance of time there might be as many equally good blockchain ledger versions as the number of miners. In addition, a peer cannot accurately estimate the number of currently active peers in the network. Consequently, it cannot determine if its local version of the blockchain ledger is accepted by the majority. 

Since a rational miner can convincingly neither deduce if its own attempt to establish a network-wide consensus about irrefutable blockchain ledger state will be successful nor deduce whether it will be worthwhile to throw away its own equally good local ledger state in response to a request from a neighbor, the only rational behavior is to keep lengthening its own local ledger version. In the worst case, all miners will emulate the same behavior and the only consented state will be the genesis state and no transaction can ever be declared final. Hence halting block mining is essential for forcing a ledger state consensus.             
\end{proof}                

At what state of the blockchain ledger the next checkpoint consensus should be attempted must be known to all network peers beforehand; otherwise they will not know when to halt in the absence of a reliable broadcast mechanism. A simple way to achieve this is to make checkpoints periodical to the length of the blockchain or the total amount of PoW. Another alternative is to decide the next checkpoint time as part of the consensus establishment for the current checkpoint.    

An important property of transaction finality in a PoW mining based blockchain network is that given it is a requirement, the rational behavior is to halt and establish a checkpoint at intended time before attempting further progress in advancing the blockchain. This is explained in the following lemma (proof omitted):

\begin{lemma}
\label{l-rationality}
The only rational behavior is to collaborate on a checkpoint consensus when a miner's local blockchain ledger approaches the checkpoint state. \qed       
\end{lemma}

Lemma \ref{l-halt} and \ref{l-rationality} suggest that a reasonable strategy to achieve transaction finality in a PoW mining based blockchain network is to periodically alternate between a block mining and a checkpoint protocol. However, the checkpoint protocol should be designed with care to avoid scalability and fairness issues during its execution and to avoid introducing security, and censorship issues in the block mining process during the protocol switching.     

