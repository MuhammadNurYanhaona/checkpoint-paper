\section{Checkpoint Algorithm}
\label{s-algorithm}
The general description of the checkpoint protocol is as follows:
\begin{enumerate}
\item If a miner, $f_i$, reaches the checkpoint candidacy state, it creates a voting token $t_{f_i}(0)$ (0 representing the token delegation depth) from its latest heartbeat acknowledgement that proves its  $BS_{f_i}^t$.
\item For each neighbor peer $p$, $f_i$ creates a time-stamped sub-token $t_{f_i}^{p}(1)$ dedicated for $p$ and requests a vote. This initiates a checkpoint consensus voting round.
\item A network peer $p$ evaluates all voting requests, $t_{f_i}^{p}(d_{f_i})$s, it has received, validates the blockchain ledgers of the corresponding candidates, determines voting for which candidate maximizes its own profit, then casts an encoded vote for that candidate, $f_c$, by sending a payload with its next heartbeat message to the support service. Peer $p$ receives a vote acceptance acknowledgement $VA_p^t$ in return.
\item A front-runner miner $f_i$ makes its own encoded vote from $t_{f_i}(0)$ and registers the vote after it receives acceptance notifications from some neighbors or after a maximum waiting time.
\item Any peer $p$ who has voted keeps reaching out for more lagging behind neighbors by sending them voting sub-tokens made of its own $t_{f_c}^{p}(d_{f_c})$ and $VA_p^t$.
\item Until the end of the voting round, peers can keep changing their votes as they hear of better alternative to their current choice.   
\item At the end of the voting round, the peers reveal their vote to support service by supplying the decoding key for their encoded vote with their next heartbeats.
\item If there is a single majority, the support service supplies sealing materials for the checkpoint block that the wining majority mines. The remaining others synchronize their ledgers with the majority and everyone switches back to the block mining phase.
\item If there is no single majority then some inferior candidates are filtered using a universally known, deterministic, and fair criteria. A new voting round begins with fewer candidates. The cycle continues in this manner until a single majority consensus is reached.                              
\end{enumerate}

Listing \ref{miner-algo} presents a redacted pseudo-code of the network peers' algorithm for the checkpoint consensus protocol. The pseudo-code does not show any error processing or malicious behavior detection.

\lstset{caption=A miner's perspective of the checkpoint protocol, label=miner-algo}
\lstinputlisting{miners-perspective.go} 

Note that a voting round is self-terminating. Each miner individually determines when to stop voting and reveal its final candidate of choice based on a support service clock counter (\textit{Line 79}). In addition, the whole consensus process is guaranteed to converge as each voting round reduces the front-runner candidates count and ensures that all honest network peers get to know about all the remaining candidates. 

Listing \ref{support-algo} presents a redacted pseudo-code of the support service side of the checkpoint algorithm. The vote revelation and error processing related logic are omitted again.

\lstset{caption=Support service's perspective of the checkpoint protocol, label=support-algo}
\lstinputlisting{support-service-perspective.go}

Any heartbeat exchange with the support service involves solving a new PoW puzzle. This is done to avoid a denial of service attack on the support service by frequent heartbeats \cite{Back02hashcash}. In addition, changing an existing vote involves solving increasingly more difficult PoW challenge (\textit{Line} \textit{25} and \textit{114}). This strategy compels rational miners to be prudent with their vote change decisions. 

The support service alternates between different heartbeat processors (\textit{Line 46, 56, 76}, and \textit{84}) based on an internal clock and the state of the ongoing checkpoint consensus process. In particular, each voting round remains open for $N - 1$ clock ticks (\textit{Line 49} and \textit{74}). Successive ticks of the internal clock should provide enough time for a blockchain ledger synchronization between a pair of interacting network peers and the \textit{activation window}, $\Omega$, of \textit{Line 58} should be large enough to allow all network peers to exchange at least one heartbeat message with some support service node.

The support service freezes the checkpoint candidate list $\Omega$ time after locally initiating the checkpoint protocol (\textit{Line 43}). Observant readers will notice that it does so without even knowing the candidates, as all votes are encoded. This operation finalizes the list of peers to be probed by lagging behind miners who did not vote yet.    
        
